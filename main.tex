\documentclass{article}
\usepackage[utf8]{inputenc}
\usepackage[T1]{fontenc}
\usepackage{tikz}
\usepackage{amsmath}
\usepackage{amssymb}
\usepackage{karnaugh-map}
\title{Assignment 1  | FPGA Lab}
\author{Robin Singh}
\date{January 2022}

\begin{document}
\maketitle

\section{Question}

A training institute intends to give scholarships to its students as per the criteria
given below :\\
• The student has excellent academic record but is financially weak.\\
                        \begin{center}     OR \end{center}  \\
• The student does not have an excellent academic record and belongs to
a backward class.\\
                             \begin{center}     OR \end{center}  \\
• The student does not have an excellent academic record and is
physically impaired.\\
The inputs are:\\
INPUTS\\
A: Has excellent academic record\\
F: Financially sound\\
C: Belongs to a backward class\\
I: Is physically impaired
(In all the above cases 1 indicates yes and 0 indicates no).\\
Output : X [1 indicates yes, 0 indicates no for all cases]\\
Draw the truth table for the inputs and outputs given above and write the SOP
expression for X(A,F,C,I).

\section{Solution}

\subsection{Truth Table}
\begin{displaymath}
\begin{array}{|c|c|c| c|c|}
% |c c|c| means that there are three columns in the table and
% a vertical bar ’|’ will be printed on the left and right borders,
% and between the second and the third columns.
% The letter ’c’ means the value will be centered within the column,
% letter ’l’, left-aligned, and ’r’, right-aligned.
\hline
A & F & C &I &\textbf{X } \\ % Use & to separate the columns
\hline % Put a horizontal line between the table header and the rest.
0 & 0 & 0 & 0 & 0 \\
0 & 0 & 0 & 1 & 1 \\
0 & 0 & 1 & 0 & 1 \\
0 & 0 & 1 & 1 & 1 \\
0 & 1 & 0 & 0 & 0 \\
0 & 1 & 0 & 1 & 1 \\
0 & 1 & 1 & 0 & 1 \\
0 & 1 & 1 & 1 & 1 \\
1 & 0 & 0 & 0 & 1 \\
1 & 0 & 0 & 1 & 1 \\
1 & 0 & 1 & 0 & 1 \\
1 & 0 & 1 & 1 & 1 \\
1 & 1 & 0 & 0 & 0 \\
1 & 1 & 0 & 1 & 0 \\
1 & 1 & 1 & 0 & 0 \\
1 & 1 & 1 & 1 & 0 \\
\hline


\end{array}
\end{displaymath}
\subsection{Karnaugh Map for given truth table}
\numberwithin{figure}{section}
\begin{figure}[h]
\centering
\begin{karnaugh-map}[4][4][1][$CD$][$$AB$$]
    \minterms{1,2,3,5,6,7,8,9,10,11}
    \maxterms{0,4,12,13,14,15}
    \implicant{8}{10}
    \implicant{1}{7}
    \implicant{3}{6}
    
    \draw[color=black, ultra thin] (0, 4) --
    node [pos=0.7, above right, anchor=south west] {$CD$} % YOU CAN CHANGE NAME OF VAR HERE, THE $X$ IS USED FOR ITALICS
    node [pos=0.7, below left, anchor=north east] {$AB$} % SAME FOR THIS
    ++(135:1);
    
\end{karnaugh-map}
\caption{Karnaugh-Map}
\label{fig:kmap}
\end{figure}

\subsection{SOP EXPRESSION}
       \begin{center}     X=A.F' +A'.C +A'.I \end{center}  \\
      
        To implement it using NAND Logic, we convert the simplified SOP expression to suite the NAND logic, which gives :

\begin{center}
 
   $F = \overline{\overline{A.\overline{F}} . \overline{\overline{A}.C} . \overline{\overline{A}.I}} .\\
   \vspace{5pt}
   
\end{center}    
\end{document}