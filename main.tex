\documentclass{article}
\usepackage[utf8]{inputenc}
\usepackage[T1]{fontenc}
\usepackage{tikz}
\usepackage{amsmath}
\usepackage{amssymb}

\title{Assignment 1  | FPGA Lab}
\author{Robin Singh}
\date{January 2022}

\begin{document}
\maketitle

\section{Question}

A training institute intends to give scholarships to its students as per the criteria
given below :\\
• The student has excellent academic record but is financially weak.\\
                        \begin{center}     OR \end{center}  \\
• The student does not have an excellent academic record and belongs to
a backward class.\\
                             \begin{center}     OR \end{center}  \\
• The student does not have an excellent academic record and is
physically impaired.\\
The inputs are:\\
INPUTS\\
A: Has excellent academic record\\
F: Financially sound\\
C: Belongs to a backward class\\
I: Is physically impaired
(In all the above cases 1 indicates yes and 0 indicates no).\\
Output : X [1 indicates yes, 0 indicates no for all cases]\\
Draw the truth table for the inputs and outputs given above and write the SOP
expression for X(A,F,C,I).

\section{Solution}

\subsection{Truth Table}
\begin{displaymath}
\begin{array}{|c c c c |c|}
% |c c|c| means that there are three columns in the table and
% a vertical bar ’|’ will be printed on the left and right borders,
% and between the second and the third columns.
% The letter ’c’ means the value will be centered within the column,
% letter ’l’, left-aligned, and ’r’, right-aligned.
A & F & C &I &\textbf{X } \\ % Use & to separate the columns
\hline % Put a horizontal line between the table header and the rest.
F & F & F & F & F \\
F & F & F & T & T \\
F & F & T & F & T \\
F & F & T & T & T \\
F & T & F & F & F \\
F & T & F & T & T \\
F & T & T & F & T \\
F & T & T & T & T \\
T & F & F & F & T \\
T & F & F & T & T \\
T & F & T & F & T \\
T & F & T & T & T \\
T & T & F & F & F \\
T & T & F & T & F \\
T & T & T & F & F \\
T & T & T & T & F \\



\end{array}
\end{displaymath}

\subsection{SOP EXPRESSION}
      \begin{align}
         X=A.F^\complement +A^\complement.C +A^\complement.I 
      \end{align}     
      
   
\end{document}